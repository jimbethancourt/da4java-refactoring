\documentclass{seal_article}
\usepackage{seal}
\usepackage{url}

\title{How to Use Logging in Evolizer Core}
\subtitle{Overview and Examples}
\author{Michael W\"{u}rsch}

\begin{document}
\maketitle

\section{Overview}
\textsc{EvolizerCore} provides sophisticated logging facilities. They are based on log4j\footnote{http://logging.apache.org/log4j/}, but provide fine-grained configuration possibilities on a per-plug-in basis. For example, it is possible to configure logging to write messages to console, to Eclipse's \texttt{ILog}, and/or to a file by simply editing \texttt{log4j.properties} without the need to change any code at all. This document describes how to use and configure logging in plug-ins which are based on the \textsc{Evolizer}-plattform in an appropriate way.

\newpage

\section{Using Logging in a Plug-In}
This section describes the steps that are necessary to use logging in an \textsc{EvolizerCore}-context.

\subsection{Implementation Details}
To use the logging facilities provided by \textsc{EvolizerCore}, some prerequisites have to be met. First of all, some plug-in dependencies must be fulfilled. Implementors require the following two modules:

\begin{itemize}
	\item{org.evolizer.util.logging}
	\item{org.apache.log4j}
\end{itemize}

The latest revisions of both of the plug-ins are located in Haydn's svn repository\footnote{https://haydn.ifi.unizh.ch/svn/evolizer\_ng/core/trunk/} and can be added to the \textbf{Required Plug-ins}-list under the \textbf{Dependencies}-tab of the PDE-view or by editing the plug-in's manifest. Either way, the following lines need to be added to MANIFEST.MF:
\\
\begin{lstlisting}[frame=tbrl, caption=META-INF/MANIFEST.MF]{manifest}
Require-Bundle: org.apache.log4j,
 org.evolizer.util.logging
\end{lstlisting}

Second, a log4j-configuration (see Section~\ref{config} for details) must be provided. Third, the plug-in's \lst{Activator}-class has to be enhanced by additional code (note that some statements, that are not directly related to logging, were left out): \\

\newpage

\lstset{language=java}
\begin{lstlisting}[frame=tbrl, caption=org.evolizer.foo.Activator -- Establishing logging for a custom plug-in]{activator}
public class Activator extends Plugin {
	static final String LOG_PROPERTIES_FILE = "config/log4j.properties";
	PluginLogManager fLogManager;

	public void start(BundleContext context) throws Exception {
		super.start(context);
		configure();
	}

	public void stop(BundleContext context) throws Exception {
		Activator.plugin = null;
		if (fLogManager != null) {
			fLogManager.shutdown();
			fLogManager = null;
		}
		super.stop(context);
	}

    public static InputStream openBundledFile(String filePath)
    										throws IOException {
            return Activator.getDefault().getBundle()
            							.getEntry(filePath).openStream();
    }
    
	public static PluginLogManager getLogManager() {
		return getDefault().fLogManager; 
	}

	private void configure() {
		try {
			InputStream propertiesInputStream
						= openBundledFile(LOG_PROPERTIES_FILE);		
			if (propertiesInputStream != null) {
				Properties props = new Properties();
				props.load(propertiesInputStream);
				propertiesInputStream.close();
				fLogManager = new PluginLogManager(this, props);
			}
			propertiesInputStream.close();
		} catch (Exception e) { /*Exception-handling*/ }
	}
}
\end{lstlisting}

Clients can now retrieve a logger-instance from the \lst{Activator} by invoking the following code:\\

\lstset{language=java}
\begin{lstlisting}[frame=tbrl, caption=org.evolizer.foo.Bar -- Getting a logger-instance]{bar}
public class Bar {
	Logger logger
		= Activator.getLogManager().getLogger(Bar.class.getName());
}
\end{lstlisting}

Note that each plug-in receives its own logger hierarchy and configuration. This opens the possibility to turn of logging of one plug-in completely, while \eg printing detailed logging information of a second plug-in to console. This is very convenient, since it reduces information overload significantly.

To add a message to the log, several methods can be used, depending on the severity of the logging cause:\\

\lstset{language=java}
\begin{lstlisting}[frame=tbrl, caption=org.evolizer.foo.Bar -- Logging]{bar}
public class Bar {
	...
	void doSomething(){
		logger.debug("this is a debugging statement");
		logger.warn("this is a warning statement");
		logger.info("this is an info statement");
		logger.error("this is an error statement");
	}
	...
}
\end{lstlisting}

The next section discusses the circumstances under which a certain logging level may be appropriate.

\newpage

\subsection{Guidelines - Under construction}

This section defines logging conventions with the following goals in mind:

\begin{itemize}
	\item{User and developer friendly logging}
	\item{Expressiveness of logging statements}
	\item{Uniformity of logger use}
\end{itemize}

Not every log statement that is interesting for a developer might be useful for the user. Severity levels like debug, warn, info, and error provide a convenient way to handle this issue:

\begin{enumerate}

	\item{Use \lst{logger.debug()} to log information that is not usefull for the user, but only for the developer.}

	\item{Use \lst{logger.warn()} and \lst{logger.info()} for information that the user should be noticed of. Use these methods wisely, since they will be sent \eg to Eclipse's log and might clutter the Log View.}

	\item{Use \lst{logger.error()} for example when an exception occurs and the normal program path cannot be continued.}

\end{enumerate}

\newpage

\section{Configuration Example}
\label{config}

Configuration can be done without changing a single line of code. Instead, each plug-in needs a folder \texttt{config/} containing a file \texttt{log4j.properties}. The following listing shows an example configuration:\\

\lstset{language=bash}
\begin{lstlisting}[frame=tbrl, caption=log4j.properties -- Configuration]{properties}
# Set root logger level to [LEVEL] and its appender to [APPENDER].
# Suggested levels are: DEBUG, WARN, INFO, ERROR, OFF.
log4j.rootLogger=DEBUG, A1, A2, R1

#Plug-in-specific loggers
log4j.logger.org.evolizer.foo.Bar=ERROR, A1

# A1 is set to be a ConsoleAppender.
log4j.appender.A1=org.apache.log4j.ConsoleAppender

# A2 is set to be a EclipseLogAppender
log4j.appender.A2=org.evolizer.util.logging.EclipseLogAppender
log4j.appender.A2.verbose=true
log4j.appender.A2.layout=org.apache.log4j.PatternLayout
log4j.appender.A2.layout.ConversionPattern=%p %t %c - %m%n

# A1 uses PatternLayout.
log4j.appender.A1.layout=org.apache.log4j.PatternLayout
log4j.appender.A1.layout.ConversionPattern=%-4r [%t] %-5p %c %x - %m%n

# R1 is set to be a RollingFileAppender
log4j.appender.R1=org.apache.log4j.RollingFileAppender

log4j.appender.R1.File=myfile.log
log4j.appender.R1.MaxFileSize=10MB
# Keep one backup file
log4j.appender.R1.MaxBackupIndex=100

log4j.appender.R1.layout=org.apache.log4j.PatternLayout
log4j.appender.R1.layout.ConversionPattern=%-4r [%t] %-5p %c %x - %m%n
}
\end{lstlisting}

\newpage

The properties listed above configure the logging system to behave as follows:

\begin{enumerate}
\item{All log messages will be sent to the appenders, since the logger level is set to \texttt{DEBUG} \\-- except for the class \lst{org.evolizer.foo.Bar}, whose logger level is set to \texttt{ERROR}.}
\item{All log statements are written to console (appender A1), to the Eclipse log (appender A2), and to a file called \texttt{myfile.log}. Again, \lst{org.evolizer.foo.Bar} is an exception since its errors are only sent to console.}
\end{enumerate}

This configuration will lead to many entries in Eclipse's log, which is very likely to be undesired behavior. Changing\\

\noindent \lst{log4j.appender.A2.verbose=true}\\

\noindent to\\

\noindent \lst{log4j.appender.A2.verbose=false}\\

\noindent takes care of this issue by telling the EclipseLogAppender to handle only errors. See the log4j-documentation for details on appenders and layouts.

\bibliographystyle{abbrv}
\bibliography{bib_template}

\end{document}
